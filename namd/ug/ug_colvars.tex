\section{Collective Variable-based Calculations (Colvars)}
\label{section:colvars}

\subsection{Overview of Colvars module and links to documentation}

The collective variables module (Colvars) is an independently-developed library for enhanced sampling simulations, licensed under the GNU lesser general public license (\url{https://www.gnu.org/licenses/lgpl-3.0.en.html}) and distributed together with NAMD.
Full documentation and examples for this software, including any syntax and features specific to this version of NAMD, can be accessed at the following links:
\begin{itemize}
\item (HTML) \url{https://colvars.github.io/namd-3.0/colvars-refman-namd.html}
\item (PDF) \url{https://colvars.github.io/namd-3.0/colvars-refman-namd.pdf}
\end{itemize}
\noindent{}See in particular the section regarding compatibility between versions of NAMD:\\
\url{https://colvars.github.io/namd-3.0/colvars-refman-namd.html#sec:colvars_config_changes}

When using the Colvars module please cite the following publication, alongside other publications for specific features as listed in the code's documentation and in the usage summary printed when running a Colvars-enabled NAMD simulation.

\begin{quote}
  G.\ Fiorin, M.\ L.\ Klein, and J.\ H\'enin, {\it Molecular Physics} {\bf 111} (22-23), 3345-3362 (2013). \url{https://doi.org/10.1080/00268976.2013.813594}
\end{quote}

Please ask any usage questions through the NAMD mailing list, and any development questions through the Colvars GitHub repository at \url{https://github.com/Colvars/colvars}.
